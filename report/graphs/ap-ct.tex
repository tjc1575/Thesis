\documentclass{standalone}[]

\usepackage{tikz}
\usepackage{pgfplots}
\usepgfplotslibrary{statistics}
\usetikzlibrary{pgfplots.statistics}
\usepackage{etoolbox}

\newtoggle{label}
\togglefalse{label}

\pgfplotsset{compat=1.12}

\begin{document}
\begin{tikzpicture}
\begin{axis}[
	ymajorgrids,
	yminorgrids,
	ytick={0,10,...,100},
	xtick={1,2,3,4},
	xticklabel style={align=center},
	xticklabels={RF\\R-M, RF\\M-R, ANN\\R-M, ANN\\M-R},
	title=All Participants - Cross Task, 
	ylabel=Accuracy (\%),
	ymin=0,
	ybar,
	bar shift=0pt,
	bar width=12pt,
	\iftoggle{label}{
		nodes near coords,
    		nodes near coords align={vertical},
	}{}
	]
	
	\addplot+
	 coordinates { 
		(1,50.58) 
		(2,49.39) 
	};
	
	\addplot+
	 coordinates { 
		(3,53.43) 
		(4,56.24) 
	};
	
\end{axis}
\end{tikzpicture}

\end{document}
















